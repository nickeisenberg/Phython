\documentclass{article}
\usepackage{amssymb, latexsym, amsmath, graphics, fullpage, epsfig, amsthm, relsize, pgf, tikz, amsfonts, makeidx, latexsym, ifthen, hyperref, calc}
%\usepackage{eucal} this is a different font than \mathcal{•}
\usetikzlibrary{arrows}
\newtheorem{theorem}{Theorem}[section]
\newtheorem{remark}{Remark}[section]
\newtheorem{corollary}{Corollary}[section]
\newtheorem{lemma}{Lemma}[section]
\newtheorem{assumptions}{Assumptions}[section]
\newtheorem{proposition}{Proposition}
\newtheorem{definition}{Definition}
\newtheorem{notation}{Notation}
\usepackage{mathrsfs}
\usepackage[shortlabels]{enumitem}
\usepackage{tikz}
\usepackage{amsmath}
\usetikzlibrary{arrows}
\usetikzlibrary{shadows.blur}
\usepackage{pgfplots}
\usepackage{mwe} % For dummy images
\usepackage{subcaption}
\usepackage{multirow}
\usepackage{dcolumn}
\newcolumntype{2}{D{.}{}{2.0}}
\usepackage{multicol}
\numberwithin{equation}{section}
\usepackage{showlabels}

\newcommand{\lip}{\textup{Lip}_b}
\newcommand{\liph}{\text{Lip}_{\hat\rho}}
\newcommand{\R}{\mathbb{R}}
\newcommand{\E}{\mathbb{E}}
\newcommand{\Norm}[1]{\left\|  #1   \right\|}
\newcommand{\ud}{\ensuremath{\mathrm{d}}}

\usepackage{lipsum}%% a garbage package you don't need except to create examples.
\usepackage{fancyhdr}
\pagestyle{fancy}
%\lhead{\Huge Version A}
%\rhead{\thepage}
\cfoot{}
\renewcommand{\headrulewidth}{0pt}

\usepackage[headheight = .5in, headsep = \baselineskip, top = 1in, left = 1in, right = 1in, textwidth = 7.52 in]{geometry}
\lhead{ \parbox[][\headheight][t]{5cm}{\textbf{}}}
\rhead{\parbox[][\headheight][t]{2cm}{\raggedleft Page\,\thepage{} of \pageref{LastPage}}}
\usepackage{lastpage}

\usepackage{tikz}
\usepackage{pgfplots}
\pgfplotsset{compat=newest}
\usetikzlibrary{decorations.markings}

\allowdisplaybreaks


\begin{document}
\begin{center}
	A back test to see how accurate the black scholes formula is.
\end{center}

It is known that the price of an American and European call on a non-dividend paying stock is the same. For that reason we should be able to use the Black-Scholes formula to price an American call on an arbitrary non-dividend paying stock. We will need the following prelims for the set up of the Black-Scholes formula 

\section{prelims}
\subsection{geometric brownian motion}
Denote the  price of a stock at time $t$ by $S_t$. We assume that the price of the stock follows a geometric Brownian motion:
	\begin{equation} \label{E: geo_b}
		dS_t = \mu S_t \ud t + \sigma S_t \ud B_t
	\end{equation}
where $B_t$ is a standard Brownian motion. Through the use of Ito's formula, it can be shown that the solution to SDE \eqref{E: geo_b} is given by the following:
	\begin{equation} \label{E: geo_b_sol}
		S_t = S_0 \exp\left( \left( \mu - \frac{1}{2}\sigma^2 \right)t + \sigma B_t \right).
	\end{equation} 
Note that this heuristically suggests that the average rate of return on the stock is $\mu$ and that the variance is $\sigma^2$. This is suggested by the following:
	\begin{equation} \label{E: geo_b_roc}
		\frac{dS_t}{S_t} = \mu \ud t + \sigma \ud B_t.
	\end{equation}
Suppose we denote by $R_x$, the rate of return of $S_t$ over the period $[t, t+x]$. In other words,
	\[
		R_x = \frac{S_{t+x} + S_t}{S_t}.
	\]
Thus on the interval $[t, t+x]$, \eqref{E: geo_b_roc} becomes 
	\begin{equation}
		R_x = \mu \cdot x + \sigma (B_{t + x} - B_t),
	\end{equation}
which implies that,
	\begin{equation}
		\mathbb{E}\left(\frac{R_x}{x}\right) = \mu
	\end{equation}
since the expected value of a Brownian motion, $\mathbb{E}(B_t) = 0$. This suggestes that the average rate of change of $R$ over $[t, t + x]$ is $\mu$. Furthermore, since $\mathbb{E}(B_t^2) = t$, then 
	\[
		V(R_x) = \mathbb{E}(R_x^2) - \mathbb{E}(R_x)^2 = \mu^2 x^2 + \sigma^2 (x) - \mu^2 x^2 = \sigma^2 x,
	\]
and thus,
	\[
		\frac{	V(R_x) }{x} = \sigma^2.
	\]
\subsection{self-financing strategy}
Consider a bond with price at time $t$ given by $\beta_t$ with risk free rate $r$. Next, consider a portfolio consisting only of shares of one stock and bonds. Let $a_t$ be the amount of shares at time $t$ and $b_t$ be the amount of bonds at time $t$. Then the values of the portfolio at time $t$ is given by,
	\[
		V_t := a_t S_t + b_t \beta_t,
	\] 
where the initial value of the portfolio is given by $V_0$. Consider of the period $[0,t]$, the capital gains of the stock given by,
	\[
		a_t(S_t - S_0).
	\]
Next, suppose that we partition out interval $[0,t]$ by $0 = t_0, \cdots, t_i, \cdots, t_{n} = t$ and that we actively manage our stock holdings over this time period such that out current stock holding is given by the following process,
	\begin{equation} \label{E: holdings}
		a_t = \sum_{i=0}^n a_{t_i} \textbf{1}_{[t_{i-1}, t_i]}(t).
	\end{equation}
Then the capital gains take the following form,
	\begin{equation} \label{E: captiol_gains}
		\sum_{i=0}^n a_{t_i}(S_{t_i} - S_{t_{i-1}}),
	\end{equation}
which is just the stochastic integral of \eqref{E: holdings}. In other words, \eqref{E: captiol_gains} can be written as the following:
	\begin{equation}
		\int_0^T a_t \ud S_t = \mu \int_0^T a_t S_t\ud t + \sigma \int_0^t a_t S_t \ud B_t.
	\end{equation}
The same can be done with the bonds and $\int_0^T b_t \ud \beta_t$ can be defined equivalently.

With this, we can define a self financing strategy as a pair $(a, b)$ such that 
	\begin{equation}
		a_t S_t + b_t \beta_t = a_0 S_0 + b_0 \beta_0 + \int_0^T a_t \ud S_t + \int_0^T b_t \ud \beta_t.
	\end{equation}
In other words, the value at time $t$ is equal to the initial value of the portfolio plus all capital gains.

\section{Black-Scholes}
A European call option on a stock that does not pay any dividends that has a strike, $K$, and expires at time $T$ will pay $(S_t - K)^+$ at time $T$. The fair value of this option is defined to be the amount of money that can be invested into a self financing strategy at time $0$ that at time $T$ will be worth $(S_t - K)^+$. This fair value only depends on the price $S_0$ and the time to expire $T$. It is derived to be the following:
	\begin{equation}
		f(S_0, T) = S_0 \phi(g(S_0, T)) - Ke^{-rt}\phi(h(S_0, T)),
	\end{equation}
where $\phi(t)$ is the standard normal cumulative distribution and
	\[
		g(x, t) = \left[ \ln(x / K) + \left( r + \frac{1}{2} \sigma^2 \right)t \right] / \sigma \sqrt{t},
	\]
	\[
		h(x,t) = g(x, t) - \sigma \sqrt{t}.
	\]
Furthermore, it can be shown that the amount of shares owned $a_t$ and bonds owned $b_t$ are given by the following:
	\[
		a_t = \frac{\partial}{\partial x} f(S_t, T-t) \quad \text{and} \quad b_t = \frac{f(S_t, T-t) - a_tS_t}{\beta_t}.
	\]
	
\section{back-test}
Historical end of day option prices are freely available though Polygon I believe. Here is what we can try.
	\begin{enumerate}
		\item Pick a non dividend paying stock, AMZN for example.
		\item Pick a date in the past, 11-1-2022. This will be time $t=0$.
		\item Pick a call in the future. There was a 95 strike call expiring on 12-2-2022.
		\item Find the price at $t=0$. According to \href{https://polygon.io/docs/options/get_v2_aggs_ticker__optionsticker__range__multiplier___timespan___from___to}{Polygon}, this call opened at $\$ 7$ and closed at $\$6.04$.
		\item Since we are back testing, $S_t$ is not a random variable, we can treat it as a deterministic sequence and then calculate $a_t$ and $b_t$ explicitly. To do this we can partition the time interval 2022-11-1 to 2022-12-2 and then do find the fair price of the option discretely. We can see if the fair price is higher or lower than the close price of the option on 2022-11-1. We also run a monte carlo simulation for AMZN by simulation a geometric brownian motion with $\mu$ and $\sigma$ correctly chosen. With this simulations, we can calculate the fair price of the option and see how theses prices compare. 
	\end{enumerate}
\end{document}
